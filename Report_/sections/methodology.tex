\section{Methodology}

In training an agent to play Blackjack, an iterative Q-Learning approach was taken. Watkins' Q-Learning aims to learn the optimal 'q-value' for given state-action pairs in an environment, i.e., the respective value of making a certain move in a certain environmental state. This approach was chosen because it is a model-free, value-based, off-policy algorithm. Model free means that , value based means that  , and off-policy means that   .
WHAT DOES THIS MEAN!  
Q-learning relies on Markov Decision Processes, which enable the probabilistic nature of Blackjack to be realized by the model. MORE Q-LEARNING + MDP

The Bellman's equation for Q-Learning is defined as
\begin{multline} \label{eq:Bellman}
    Q_{new}(s,a)=Q_{old}(s,a)+\\\alpha(\underbrace{R(s,a)+\gamma MaxQ(s',a')-Q_{old}(s,a)}_{\mathclap{\text{Temporal Difference}}})
\end{multline}
Where \( s, a \) are the current state and action,  \( s', a' \) are the next state and action, \( \alpha \) is the learning rate, \( \gamma \) is the discount factor, \( R(s,a) \) is the reward received after taking action \( a \) in state \( s \), and \( Q(s,a) \) is the q-value for the state-action pair. The highlighted temporal difference is 
used to update an estimate based on other estimates, without waiting for a final outcome (known as bootstrapping) \cite{10.5555/3312046}. The learning rate \(\alpha\), due to the probabilistic nature of the game, decayed throughout training. It took the equation, 
\begin{equation}
    \alpha = \alpha_{min} + (\alpha_{max} - \alpha_{min})e^{-\frac{1}{dt}}
\end{equation}
For respective values for the minimum and maximum learning rate, decay rate \(d\), and number of iterations over the training process \(t\). The reward \(R(s,a)\), passed to the agent to drive learning, was defined by
\begin{equation} \label{reward}
    R(s,a) = S_{new}^2 - (S_{old}^2 - S_{new}^2)\delta_{Ace}
\end{equation}
Where \(\delta_{Ace}=1\) when an ace drops in value from 11 to 1, and \(\delta_{Ace}=0\) otherwise, accounting for the disadvantage.

\smallskip
Q-tables encapsulate the state-action pairs, entirely defining all possible moves in the system with their respective Q-value. Each action taken recalculates respective state-action Q-values using equation \ref{eq:Bellman}. This converges towards a terminal value which, for probabilistic situations like this, is the expected ... see notes

From this, the optimal policy - set of moves - was determined using
\begin{equation}
    q_*(s,a) = \max_{\pi}q_{\pi}(s,a)\text{.}
\end{equation}

\smallskip
Since an agent follows the optimal policy where possible, exploiting  and not exploring new information, an Epsilon-greedy algorithm was employed. This intends to explore less taken state-action avenues \(X\%\) of the time. 

\subsection{Infinite}

In the "infinite" situation, the probability of each card being drawn is equal, so retaining prior knowledge of cards drawn poses no advantage. Therefore, the Q-table consisted of the dimensions; 'score' (2$\rightarrow $ 20), 'ace valued at 11 in hand' (Yes or No), and 'action' (Hit or Stick).


\subsection{Finite}

In the 'finite' situation, cards were drawn from a pile of finite number, meaning that the probability of drawing respective cards changed as the game progressed. This posed a new challenge which could ideally be solved by providing the agent with all previously dealt cards, from which it could learn to predict the probabilities of newly dealt cards, and so, how risk-adverse it should play. However, doing so would be at great computational cost, where the Q-table would need to incorporate the dimensions previously described, in addition to some combination of previously dealt cards. 

To strike a balance between accuracy and potential advantage, the probability of 

... consider other methods (like card counting)

? average of recived cards???

find a reference that justifies using a decreasing alpha for probabilistic problems. 
