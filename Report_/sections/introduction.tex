\section{Introduction}

Blackjack, also known as twenty-one, is the most widely played casino game in the world, largely due to its simple game structure in which a player attempts to get the highest score by drawing cards from a deck. With its long history, various strategies for increasing a players chance of winning, such as card counting, are commonplace. Whilst an optimum strategy for blackjack has been known by statisticians for decades, by drawing on the expected value of future cards \cite{Baldwin01091956}, training an intelligent agent to learn the optimum strategy is an approach not as often taken. Using intelligent agents like this is known within machine learning communities as 'reinforcement learning'. The advantage of taking a reinforcement learning approach in this circumstance is rooted in the stochastic nature of blackjack, in which long term reward is prioritized. Additionally, its theorem proving nature offers extra insight, i.e., it can validate or indicate mathematical frameworks behind a system \cite{bidi2023reinforcementlearningcontroltheory}. 

\smallskip
Reinforcement Learning is a subfield of machine learning concerned with teaching an 'agent' to find an optimal set of moves - optimal policy - in the context of a wider system. The foundational components compose a policy - mapping possible states to possible actions, a reward signal - defining the agent's goal, an environment - which the agent interacts with, and a value function - indicating the long-term desirability of a sequence of states. A reinforcement learning task is considered a Markov Decision Processes (MDP) if it satisfies the Markov property, defined as the exclusive reliance of the likelihood of changing to a specific state on the present state and elapsed time, and not on previous states. MDPs are used in sequential decision-making in probabilistic systems \cite{10.5555/3312046}.     WRITE MORE ABOUT PROBABILISTIC REINFORCEMENT LEARNING. 

%Probabilistic reinforcement learning extends traditional reinforcement learning by incorporating uncertainty into decision-making processes through probabilistic transitions between states and rewards. This approach is particularly useful in environments where outcomes are uncertain or stochastic. In the context of Markov Decision Processes (MDPs), PRL naturally fits because MDPs inherently involve probabilities: they model state transitions and reward distributions, allowing for optimal policies that account for these uncertainties. (GENERATED)


%Given a finite MDP with state \(s\) and action \(a\), the probability of each following state and reward pair is given by
% \begin{multline}
%     p(a',r|s,a)=\\\text{Pr}\{S_{t+1}=s',R_{t+1}=r|S_t=s,A_t=a\}.
% \end{multline}
% Which entirely specifies the dynamics of the system.

\smallskip
There are many variations to Blackjack, typically involving multiple players and a dealer, but for the purposes of this project a stylized version with a single player and a passive dealer was played. The game uses a standard deck of cards with numerical values equal to their number for cards 2-10, and equal to 10 for Jacks, Queens, and Kings. Aces are valued at 11 unless the sum of the cards in hand exceeds 21, in which case they are valued at 1. The sequence of play is as follows:
\begin{enumerate}
    \item A card is dealt to the player with value \(C_1\).
    \item For \(n\) iterations, or until a total score of 21 is exceeded, the player can make one of two choices;
    \begin{enumerate} 
        \item Stick, and end the game.
        \item Hit, and receive another card with value \(C_{n+1}\).
    \end{enumerate}
    \item The final score is calculated using
        \begin{equation} \label{eq: Score}
            S= 
            \begin{cases}
                (\sum C_n)^2 & \text{if } \sum C_n \le 21\\
                0            & \text{if } \sum C_n >   21\\
            \end{cases}
        \end{equation}
\end{enumerate}
\smallskip
To approach this problem, two situations were considered. Infinite, in which the pile of cards being drawn from is infinite, meaning the probability of each card being drawn is equal, and finite, in which the pile of cards being drawn from is finite, meaning unequal probabilities. These two approaches were taken to provide a broad spectrum of results. This paper details the methodology taken for both problem situations, the results of each in context of an optimal result, and a conclusion on the efficacy of this approach.